
% | LaTeX Fonte - @autor: Medeiros, F. E. O - 2018.07.13 

% | COMANDOS CRIADOS

\newcommand{\eq}[1]{(\ref{#1})} % referenciador de equações no modo texto
\newcommand{\fig}[1]{{\ref{#1}}} % referenciador de equações no modo texto
\newcommand{\tab}[1]{{\ref{#1}}} % referenciador de equações no modo texto
% ---------------------
\newcommand{\indexar}[1]{{#1}\index{#1}} % indexação simplificada para Glossário
\newcommand{\fn}[1]{$^($\footnote{$^{)}$ #1.}$^)$} % nota de roda pé estilizada
\newcommand{\blankpage}{\ \newpage} % cria página em branco
% ---------------------
\newcommand{\para}{\indent}	% tabulação em modo texto(parágrafo)
\newcommand{\rec}{\noindent} % recuo em modo texto(recuo de parágrafo)
\newcommand{\asp}[1]{``#1''{}} % abre/fecha aspas
% ---------------------
\newcommand{\bhat}[1]{\boldsymbol{\hat{\mathrm{#1}}}} % vetor unitário ("chapéu")
\newcommand{\bvec}[1]{\boldsymbol{\mathrm{#1}}}	% vetor estilo "negrito"
\newcommand{\bmat}[1]{\boldsymbol{\mathbb{#1}}}	% matriz estilizada (aceita somente UPPERCASE)
\newcommand{\ten}[1]{\boldsymbol{[\mathrm{#1}]}} % tensor estilo "colchetes"
\newcommand{\bra}[1]{\langle #1 |} % bra
\newcommand{\ket}[1]{| #1 \rangle} % ket
\newcommand{\bk}[2]{\langle #1 | #2 \rangle} % bra-ket completo
\newcommand{\isum}[2]{\sum^{#1}_{#2}} % somatório simplificando índice inferior e superior 
% ---------------------
\newcommand{\seta}{\qquad\Rightarrow\qquad}	% seta para direita em modo matemático
\newcommand{\Seta}{\qquad\Rightarrow\qquad}	% seta bruta para direita em modo matemático
% --------------------- 
\newcommand{\uni}[1]{\mathrm{#1}} % unidade no ambiente matemático ("colado")
\newcommand{\suni}[1]{\ \mathrm{#1}} % unidade no ambiente matemático ("separado")
\newcommand{\puni}[1]{\atp{\mathrm{#1}}} % unidade no ambiente matemático ("entre parênteses")
\newcommand{\graus}{^{\circ}} % unidade de grau (º)
% ---------------------
\newcommand{\Grad}[1]{\boldsymbol{\nabla} #1} % operador Gradiente
\newcommand{\Div}[1]{\boldsymbol{\nabla} \cdot #1} % operador Divergente
\newcommand{\Rot}[1]{\boldsymbol{\nabla} \times #1} % operador Rotacional
\newcommand{\Lap}[1]{\nabla^2 #1} % operador Laplaciano
% ---------------------
\newcommand{\colog}{\mathrm{colog}}	% cologarítimo
\newcommand{\res}{\mathrm{res}}	% resíduo
\newcommand{\cte}{\mathrm{cte}}	% constante
% ---------------------
\newcommand{\arccsc}{\mathrm{\:arccsc\:}} % função  arcocossecante
\newcommand{\arcsec}{\mathrm{\:arcsec\:}} % função  arcocosecante
\newcommand{\arccot}{\mathrm{\:arccot\:}} % função  arcotangente
\newcommand{\csch}{\mathrm{\:csch\:}} % função  seno hipebólico
\newcommand{\sech}{\mathrm{\:sech\:}} % função  cosseno hipebólico
\newcommand{\arccosh}{\mathrm{\:arccosh\:}}	% função  arcocosseno hipebólico
\newcommand{\arcsinh}{\mathrm{\:arcsinh\:}}	% função  arcoseno hipebólico
\newcommand{\arctanh}{\mathrm{\:arctanh\:}}	% função  arcotangente hipebólico
\newcommand{\arccsch}{\mathrm{\:arccsch\:}}	% função  arcocossecante hipebólico
\newcommand{\arcsech}{\mathrm{\:arcsech\:}}	% função  arcosecante hipebólico
\newcommand{\arccoth}{\mathrm{\:arccoth\:}}	% função  arcocotangente hipebólico
% ---------------------
\newcommand{\atp}[1]{\left( #1 \right)}	% parenteses tamanho automático
\newcommand{\atc}[1]{\left[ #1 \right]}	% colchetes tamanho automático
\newcommand{\ats}[1]{\left\{ #1 \right\}} % chaves tamanho automático
\newcommand{\ath}[1]{\left| #1 \right|}	% módulo tamanho automático
\newcommand{\atx}[1]{\left\langle #1 \right\rangle}	% média temporal tamanho automático
\newcommand{\medt}[1]{\langle #1 \rangle} % média temporal in-line
% ---------------------
\renewcommand{\d}{\mathrm{d}} % operador diferencial
\newcommand{\drac}[2]{\frac{\mathrm{d} {#1}}{\mathrm{d} {#2}}} % fração diferencial
\newcommand{\ddrac}[2]{\frac{\mathrm{d}^{2} {#1}}{\mathrm{d} {#2}^{2}}}	% fração diferencial de segunda ordem
\newcommand{\prac}[2]{\frac{\partial {#1}}{\partial {#2}}} % fração diferencial parcial
\newcommand{\pprac}[2]{\frac{\partial^{2} {#1}}{\partial {#2}^{2}}}	% fração diferencial parcial de segunda ordem
% ---------------------
\newcommand{\fem}{\mathcal{E}} % força eletro-motriz
\newcommand{\lag}{\mathcal{L}} % lagrangeano e transformada de laplace
\newcommand{\ham}{\mathcal{H}} % hamiltoniano
\newcommand{\I}{\mathcal{I}} % símbolo para intensidade luminosa
\newcommand{\R}{\mathcal{R}}
\newcommand{\T}{\mathcal{T}}
\newcommand{\e}{\epsilon} % atalho para letra grega epsilon
\newcommand{\ve}{\varepsilon} % atalho para letra grega epsilon estilizada
\newcommand{\bnu}{\bar{\nu}} % símbolo para refletividade
\newcommand{\f}{\varphi} % função f qualquer
\newcommand{\g}{\psi} % função g qualquer
\newcommand{\transp}{\mathrm{t}} % função transposta
\newcommand{\dpe}{\mathrm{s}}
\newcommand{\dgt}[1]{\fbox{\texttt{#1}}}
%-----------------------------------------------------------------------------------------------------------------------------------------------------
\newcommand{\tline}{\hline \hline \hline \hline}

\newcommand{\vetl}{{\rule{1.5mm}{.3mm}}}
\newcommand{\vetc}{{\:\rule{.3mm}{1.5mm}}}
\newcommand{\angstrom}{\mbox{\normalfont\AA}}
