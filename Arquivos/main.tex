% -----------------------------------------
% Fonte LaTeX - @author: Medeiros, F. E. O.
% -----------------------------------------

% ALTERE ESSE DOCUMENTO DE ACORDO COM AS INSTRUÇÕES EM CADA LINHA

\documentclass[12pt, oneside]{book} % Configura formatação do documento: 'TIPO LIVRO'
\usepackage{pacotes} % Carrega pacotes essenciais (NÂO MODIFICAR O ARQUIVO 'pacotes.sty')
\usepackage[Conny]{fncychap} % Configura estilo de formatação do layout: (ver opções) -> Conny, Sonny, Lenny, Glenn, Renje, Bjarne, Bjornstrup etc.
% -----------------------------------------
% Fonte LaTeX - @author: Medeiros, F. E. O.
% -----------------------------------------

% NÃO ALTERAR ESSE DOCUMENTO A MENOS QUE SAIBA O QUE ESTEJA FAZENDO!!!
% NÃO ALTERAR ESSE DOCUMENTO A MENOS QUE SAIBA O QUE ESTEJA FAZENDO!!!
% NÃO ALTERAR ESSE DOCUMENTO A MENOS QUE SAIBA O QUE ESTEJA FAZENDO!!!
% NÃO ALTERAR ESSE DOCUMENTO A MENOS QUE SAIBA O QUE ESTEJA FAZENDO!!!
% NÃO ALTERAR ESSE DOCUMENTO A MENOS QUE SAIBA O QUE ESTEJA FAZENDO!!!

% === CONFIGURAÇÃO AVANÇADA DE PÁGINA
\geometry{a4paper}
\setlength{\hoffset}{-1in} 			% 1" + margem esquerda absoluta
\setlength{\voffset}{-1in} 			% 1" + margem superior absoluta
\setlength{\oddsidemargin}{35mm} 	% margem esquerda
\setlength{\evensidemargin}{35mm}	% margem direita

\setlength{\topmargin}{20mm}		% margem superior
\setlength{\headheight}{4mm}		% altura do cabeçalho (12pt)
\setlength{\headsep}{5mm}			% espaço entre o cabeçalho e o texto
\setlength{\textwidth}{150mm}		% largura total do texto
\setlength{\textheight}{235mm}		% altura total do texto
\setlength{\columnsep}{7mm}			% espaço entre colunas
\setlength{\doublerulesep}{0.1mm}
\linespread{1.5}	

\lhead{\leftmark}
\chead{}
\rhead{\thepage}
\lfoot{\manuscrito}
\cfoot{}
\rfoot{\abrvautor}

\renewcommand{\headrulewidth}{2pt}
\renewcommand{\footrulewidth}{1pt} % Carrega configuração das páginas (NÂO MODIFICAR O ARQUIVO 'pageconf.tex')
% 
% | LaTeX Fonte - @autor: Medeiros, F. E. O - 2018.07.13 

% | COMANDOS CRIADOS

\newcommand{\eq}[1]{(\ref{#1})} % referenciador de equações no modo texto
\newcommand{\fig}[1]{{\ref{#1}}} % referenciador de equações no modo texto
\newcommand{\tab}[1]{{\ref{#1}}} % referenciador de equações no modo texto
% ---------------------
\newcommand{\indexar}[1]{{#1}\index{#1}} % indexação simplificada para Glossário
\newcommand{\fn}[1]{$^($\footnote{$^{)}$ #1.}$^)$} % nota de roda pé estilizada
\newcommand{\blankpage}{\ \newpage} % cria página em branco
% ---------------------
\newcommand{\para}{\indent}	% tabulação em modo texto(parágrafo)
\newcommand{\rec}{\noindent} % recuo em modo texto(recuo de parágrafo)
\newcommand{\asp}[1]{``#1''{}} % abre/fecha aspas
% ---------------------
\newcommand{\bhat}[1]{\boldsymbol{\hat{\mathrm{#1}}}} % vetor unitário ("chapéu")
\newcommand{\bvec}[1]{\boldsymbol{\mathrm{#1}}}	% vetor estilo "negrito"
\newcommand{\bmat}[1]{\boldsymbol{\mathbb{#1}}}	% matriz estilizada (aceita somente UPPERCASE)
\newcommand{\ten}[1]{\boldsymbol{[\mathrm{#1}]}} % tensor estilo "colchetes"
\newcommand{\bra}[1]{\langle #1 |} % bra
\newcommand{\ket}[1]{| #1 \rangle} % ket
\newcommand{\bk}[2]{\langle #1 | #2 \rangle} % bra-ket completo
\newcommand{\isum}[2]{\sum^{#1}_{#2}} % somatório simplificando índice inferior e superior 
% ---------------------
\newcommand{\seta}{\qquad\Rightarrow\qquad}	% seta para direita em modo matemático
\newcommand{\Seta}{\qquad\Rightarrow\qquad}	% seta bruta para direita em modo matemático
% --------------------- 
\newcommand{\uni}[1]{\mathrm{#1}} % unidade no ambiente matemático ("colado")
\newcommand{\suni}[1]{\ \mathrm{#1}} % unidade no ambiente matemático ("separado")
\newcommand{\puni}[1]{\atp{\mathrm{#1}}} % unidade no ambiente matemático ("entre parênteses")
\newcommand{\graus}{^{\circ}} % unidade de grau (º)
% ---------------------
\newcommand{\Grad}[1]{\boldsymbol{\nabla} #1} % operador Gradiente
\newcommand{\Div}[1]{\boldsymbol{\nabla} \cdot #1} % operador Divergente
\newcommand{\Rot}[1]{\boldsymbol{\nabla} \times #1} % operador Rotacional
\newcommand{\Lap}[1]{\nabla^2 #1} % operador Laplaciano
% ---------------------
\newcommand{\colog}{\mathrm{colog}}	% cologarítimo
\newcommand{\res}{\mathrm{res}}	% resíduo
\newcommand{\cte}{\mathrm{cte}}	% constante
% ---------------------
\newcommand{\arccsc}{\mathrm{\:arccsc\:}} % função  arcocossecante
\newcommand{\arcsec}{\mathrm{\:arcsec\:}} % função  arcocosecante
\newcommand{\arccot}{\mathrm{\:arccot\:}} % função  arcotangente
\newcommand{\csch}{\mathrm{\:csch\:}} % função  seno hipebólico
\newcommand{\sech}{\mathrm{\:sech\:}} % função  cosseno hipebólico
\newcommand{\arccosh}{\mathrm{\:arccosh\:}}	% função  arcocosseno hipebólico
\newcommand{\arcsinh}{\mathrm{\:arcsinh\:}}	% função  arcoseno hipebólico
\newcommand{\arctanh}{\mathrm{\:arctanh\:}}	% função  arcotangente hipebólico
\newcommand{\arccsch}{\mathrm{\:arccsch\:}}	% função  arcocossecante hipebólico
\newcommand{\arcsech}{\mathrm{\:arcsech\:}}	% função  arcosecante hipebólico
\newcommand{\arccoth}{\mathrm{\:arccoth\:}}	% função  arcocotangente hipebólico
% ---------------------
\newcommand{\atp}[1]{\left( #1 \right)}	% parenteses tamanho automático
\newcommand{\atc}[1]{\left[ #1 \right]}	% colchetes tamanho automático
\newcommand{\ats}[1]{\left\{ #1 \right\}} % chaves tamanho automático
\newcommand{\ath}[1]{\left| #1 \right|}	% módulo tamanho automático
\newcommand{\atx}[1]{\left\langle #1 \right\rangle}	% média temporal tamanho automático
\newcommand{\medt}[1]{\langle #1 \rangle} % média temporal in-line
% ---------------------
\renewcommand{\d}{\mathrm{d}} % operador diferencial
\newcommand{\drac}[2]{\frac{\mathrm{d} {#1}}{\mathrm{d} {#2}}} % fração diferencial
\newcommand{\ddrac}[2]{\frac{\mathrm{d}^{2} {#1}}{\mathrm{d} {#2}^{2}}}	% fração diferencial de segunda ordem
\newcommand{\prac}[2]{\frac{\partial {#1}}{\partial {#2}}} % fração diferencial parcial
\newcommand{\pprac}[2]{\frac{\partial^{2} {#1}}{\partial {#2}^{2}}}	% fração diferencial parcial de segunda ordem
% ---------------------
\newcommand{\fem}{\mathcal{E}} % força eletro-motriz
\newcommand{\lag}{\mathcal{L}} % lagrangeano e transformada de laplace
\newcommand{\ham}{\mathcal{H}} % hamiltoniano
\newcommand{\I}{\mathcal{I}} % símbolo para intensidade luminosa
\newcommand{\R}{\mathcal{R}}
\newcommand{\T}{\mathcal{T}}
\newcommand{\e}{\epsilon} % atalho para letra grega epsilon
\newcommand{\ve}{\varepsilon} % atalho para letra grega epsilon estilizada
\newcommand{\bnu}{\bar{\nu}} % símbolo para refletividade
\newcommand{\f}{\varphi} % função f qualquer
\newcommand{\g}{\psi} % função g qualquer
\newcommand{\transp}{\mathrm{t}} % função transposta
\newcommand{\dpe}{\mathrm{s}}
\newcommand{\dgt}[1]{\fbox{\texttt{#1}}}
%-----------------------------------------------------------------------------------------------------------------------------------------------------
\newcommand{\tline}{\hline \hline \hline \hline}

\newcommand{\vetl}{{\rule{1.5mm}{.3mm}}}
\newcommand{\vetc}{{\:\rule{.3mm}{1.5mm}}}
\newcommand{\angstrom}{\mbox{\normalfont\AA}}
 % Carrega comandos criados pelo usuário (OPICIONAL 'comandos.tex')


% CONFIGURAÇÃO DO MANUSCRITO
\newcommand{\brasao}{starfleet_command} %OBS.: Nome da figura (NA PASTA 'Imagens' e SEM A EXTENSÃO)
\newcommand{\universidade}{Universidade da Frota Estelar} %Ex.: Universidade Federal do Ceará
\newcommand{\centro}{Centro de Exploração Espacial} %Ex.: Centro de Ciências
\newcommand{\departamento}{Departamento de Defesa da Terra} %Ex.: Departamento de Física
\newcommand{\programa}{Programa de Pesquisa e Diplomacia} %Ex.: Programa de Pós-Graduação em Física
\newcommand{\titulo}{Aplicação da técnica telepática conhecida como Fusão Mental na transferência de energia Katra} %OBS.: Não pode passar de 3 linhas
\newcommand{\autor}{Sachn Tegai Spock} %(Nome do autor) Ex. James Tiberius Kirk
\newcommand{\abrvautor}{S. T. Spock} %(Nome abreviado do autor) Ex.: J. T. Kirk 
\newcommand{\local}{São Francisco | Califórnia} %Ex.: Fortaleza - CE
\newcommand{\manuscrito}{Tese de Doutorado} %Ex.: Tese de Doutorado
\newcommand{\graduacao}{Oficial de Ciências} %Ex.: Bacharel, Mestre, Doutor
\newcommand{\curso}{Astrofísica} %Ex.: Física, Engenharia Elétrica, Medicina 
\newcommand{\orientador}{Embaixador Sarek} %Nome do orientador
\newcommand{\datadeaprovacao}{8 de setembro de 1966} %Ex.: 8 de setembro de 1966

%BANCA EXAMINADORA
\newcommand{\presidente}{Embaixador Sarek} %Presidente da banca
\newcommand{\instituicaoI}{Central de Diplomacia da Frota Estelar} %Instituição do presidente da banca

\newcommand{\internoI}{Capitão Christopher Pike} %Membro Interno
\newcommand{\instituicaoII}{Comando Espacial da Frota Estelar} %Instituição do membro interno

\newcommand{\internoII}{Almirante Beckett Nechayev} %Membro Externo
\newcommand{\instituicaoIII}{Comando de Esquadra da Frota Estelar} %Instituição do membro externo

\newcommand{\externo}{[Integrante da banca \#4]}
\newcommand{\instituicaoIV}{[Instituição do integrante \#4]}

%CITAÇÃO, AGRADECIMENTOS, RESUMO e ABSTRACT
\newcommand{\citacao}{
	Vida longa e próspera
}
\newcommand{\autorcitacao}{Saudação Vulcana}

\newcommand{\agradecimentos}{
	Agradecimentos...
}
\newcommand{\resumo}{
	Resumo...
}
\newcommand{\palavraschave}{palavra1, palavra2, palavra3, palavra4, palavra5}

\newcommand{\abstract}{
	Abstract...
}
\newcommand{\keywords}{keyword1, keyword2, keyword3, keyword4, keyword5}

\makeindex

\begin{document}
	\frontmatter
		\pagestyle{fancy}
		% -----------------------------------------
% Fonte LaTeX - @author: Medeiros, F. E. O.
% -----------------------------------------

% NÃO ALTERAR ESSE DOCUMENTO A MENOS QUE SAIBA O QUE ESTEJA FAZENDO!!!
% NÃO ALTERAR ESSE DOCUMENTO A MENOS QUE SAIBA O QUE ESTEJA FAZENDO!!!
% NÃO ALTERAR ESSE DOCUMENTO A MENOS QUE SAIBA O QUE ESTEJA FAZENDO!!!
% NÃO ALTERAR ESSE DOCUMENTO A MENOS QUE SAIBA O QUE ESTEJA FAZENDO!!!
% NÃO ALTERAR ESSE DOCUMENTO A MENOS QUE SAIBA O QUE ESTEJA FAZENDO!!!

\thispagestyle{empty}
\begin{center}
	\includegraphics[height=41mm]{../Imagens/\brasao} \\
	{\Large \textsc{\universidade}} \\
	{\large \textsc{\centro}} \\
	{\large \textsc{\departamento}} \\
	{\large \textsc{\programa}} \\
	\vspace{35mm}
	{\LARGE \textsc{\titulo}} \\
	\vspace{60mm}
	{\large \textsc{\autor}} \\
	\vspace{10mm}
	{\large \textsc{\local}} \\
	\today
\end{center}

%-------------------------------------
\newpage
%-------------------------------------

\begin{center}
	{\LARGE \textsc{\autor}}
\end{center}

\vspace{35mm}
\noindent{\Large \textsc{\titulo}}
\vspace{40mm}

\begin{flushright}
	\begin{minipage}{80mm}
		\manuscrito{} apresentada ao \programa{} da \universidade{}, como requisito obrigatório para obtenção do título de \graduacao{} em \curso. 
		\begin{flushright} \textbf{Orientador:} \orientador \end{flushright}
	\end{minipage}
\end{flushright}
\vspace{60mm}

\begin{center}
	{\large \textsc{\local}} \\
	\today
\end{center}

%-------------------------------------
\newpage
%-------------------------------------

\begin{center}
	{\large \textsc{\autor}}
\end{center}
\vspace{5mm}

\noindent{\LARGE \textsc{\titulo}} \\
\vspace{3mm}

\begin{flushright}
	\begin{minipage}{80mm}
		\manuscrito{} apresentada ao \programa{} da \universidade{}, como requisito obrigatório para obtenção do título de \graduacao{} em \curso. 
		\begin{flushright} \textbf{Orientador:} \orientador \end{flushright}
	\end{minipage}
\end{flushright}
\vspace{5mm}

\begin{flushleft}
	\textbf{Aprovada em:} \datadeaprovacao.
\end{flushleft} \vspace{5mm}

\begin{center}
	\textsc{Banca Examinadarora} \\ 
	\vspace{10mm}
	\rule{120mm}{0.1mm} \\
	\presidente \\
	\instituicaoI \\ 
	\vspace{10mm}
	\rule{120mm}{0.1mm} \\
	\internoI \\
	\instituicaoII \\ 
	\vspace{10mm}
	\rule{120mm}{0.1mm} \\
	\internoII \\
	\instituicaoIII
\end{center}

%-------------------------------------
\newpage
%-------------------------------------

\noindent 
\vspace{180mm}
\\
\texttt{
	Dados Internacionais de Catalogação na Publicação \\
	\universidade \\
	Biblioteca Setorial de \curso
}

%-------------------------------------
\newpage
%-------------------------------------

\noindent \\
\vspace{120mm}
\begin{flushright}
	\begin{minipage}{70mm}
		``\emph{\citacao}''
		\begin{flushright}
			--- \autorcitacao
		\end{flushright}
	\end{minipage}
\end{flushright}

\chapter{Agradecimentos}
	
	\agradecimentos

\chapter{Resumo}
	\resumo \\
	\\
	\noindent \textbf{Palavras-chave:} \palavraschave.
%

%-------------------------------------
\newpage
%-------------------------------------

\chapter{Abstract}

	\abstract \\
	\\
	\noindent \textbf{Keywords:} \keywords.
%
		\tableofcontents
		\listoffigures
		\listoftables
	\mainmatter
		% Carrega os capítulos
		% \chapter{Introdução}

    \section{Lorem Ipsum}

        \lipsum

        \subsection{Cras viverra metus}

            \lipsum[1-2] \cite{LaTeX}

        \subsection{Aenean faucibus}

            \lipsum[2-3] \cite{google}

    \section{Curabitur Dictum Dravida}

        \lipsum[3-4] \cite{linux}

        \subsection{Integer sapien}

            \lipsum[4-5] \cite{html}

            \subsubsection{Dictum dravida}

                \lipsum[3] \cite{css}

    \section{Pellentesque Habitant}

        \lipsum[5-6] \cite{javascript}

        \subsection{Vestibulum luctus nibh}

            \lipsum[1-2] \cite{php}

        \subsection{Nam dui ligula}

            \lipsum[3-4] \cite{python}

        \subsection{Nulla malesuada porttitor diam}

            \lipsum[5-6] \cite{arduino}

		\chapter{Experimental}

\begin{equation}
	\uni{\Gamma_{Raman}^{300 K} = A_{1g} \oplus E_{g} \oplus 2T_{2g}}
	\label{gamma_raman_300}
\end{equation}

\begin{equation}
	\uni{\Gamma_{IR}^{300 K} = 4\uni{T_{1u}}}
	\label{gamma_ir_300}
\end{equation}

\begin{table}[ht] \centering
	\begin{tabular}{cccl} \hline \hline
	    \multicolumn{4}{l}{Cúbica - $Fm\bar{3}m$ - $O^5_h$ - $\#225$}
 	    \\ \hline
		Íon & Sítio & Simetria & Contribuição \\ \hline
		Cs & 8c & $T_d$ & $\uni{T_{1u} \oplus T_{2g}}$\\
		Ag & 4a & $O_h$ & $\uni{T_{1u}}$ \\
		Bi & 4b & $O_h$ & $\uni{T_{1u}}$ \\
		Br & 24e & $C_{4v}$ & $\uni{A_{1g} \oplus E_{g} \oplus T_{1g} \oplus T_{2g} \oplus 2T_{1u} \oplus T_{2u}}$ \\ \hline
		&  &  & $
		\uni{\Gamma_{total} = A_{1g} \oplus E_{g} \oplus T_{1g} \oplus 2T_{2g} \oplus 5T_{1u} \oplus T_{2u}} $ \\
		&  &  & $\uni{\Gamma_{acoustic} =\uni{T_{1u}}}$ \\
		&  &  & $\uni{\Gamma_{IR} = 4\uni{T_{1u}}}$ \\
 		&  &  & $\uni{\Gamma_{Raman} = A_{1g} \oplus E_{g} \oplus 2T_{2g}}$ \\
 		&  &  & $\uni{\Gamma_{silent} = T_{1g} \oplus T_{2u}}$ \\ 
 		\hline \hline
	\end{tabular}
	\caption{Teoria de grupo da peroviskita dupla $\uni{Cs_2 Ag Bi Br_6}$ na sua fase cúbica. $\uni{a = 11,264\AA}$ em $300 \uni{K}$.}
	\label{tab_cubic}
\end{table}

$$
	\begin{array}{l}
		\mbox{Transformação dos} \\
		\mbox{modos Raman}
	\end{array}
	\left\{
		\begin{array}{lll}
			\uni{A_{1g}} & \rightarrow & \uni{A_{g}} \\
			\uni{E_{g}} & \rightarrow & \uni{A_{g}} \oplus \uni{B_{g}} \\
			\uni{2T_{2g}} & \rightarrow & 2\atp{\uni{B_{g}} \oplus \uni{E_{g}}} \\
			\uni{T_{1g}} & \rightarrow & \uni{A_{g}} \oplus \uni{E_{g}} \\ \hline 
			& & \uni{3A_{g} \oplus 3B_{g} \oplus 3E_{g}}
		\end{array}
	\right.
	\label{raman_modes_transform}
$$

\begin{equation}
	\uni{\Gamma_{Raman}^{30 K} = 3A_{g} \oplus 3B_{g} \oplus 3E_{g}}
	\label{gamma_raman_30}
\end{equation}

$$
	\begin{array}{l}
		\mbox{Transformação dos} \\
		\mbox{modos infravermelhos}
	\end{array}
	\left\{
		\begin{array}{lll}
			\uni{4T_{1u}} & \rightarrow & 4\atp{\uni{A_{u}} \oplus \uni{E_{u}}} \\
			\uni{T_{2u}} & \rightarrow & \uni{B_{u}} \oplus \uni{E_{u}} \\ \hline 
			& & \uni{4A_{u} \oplus 5E_{u} \oplus \underbrace{\uni{B_{u}}}_{silent}}
		\end{array}
	\right.
	\label{IR_modes_transform}
$$

\begin{equation}
	\uni{\Gamma_{IR}^{30 K} = 4A_{u} \oplus 5E_{u}}
	\label{gamma_ir_30}
\end{equation}


\begin{table}[hb] \centering
	\begin{tabular}{cccl} \hline \hline
	    \multicolumn{4}{l}{Tetragonal - $I4/m$ - $C^5_{4h}$ - $\#87$}
	    \\ \hline
		Íon & Sítio & Simetria & Contribuição \\ \hline
		Cs & 4d & $S_{4}$ & $\uni{A_{u} \oplus B_{g} \oplus E_{g} \oplus E_{u}}$ \\
		Ag & 2a & $C_{4h}$ & $\uni{A_{u} \oplus E_{u}}$ \\
		Bi & 2b & $C_{4h}$ & $\uni{A_{u} \oplus E_{u}}$ \\
		Br1 & 8h & $C_{s}$ & $\uni{2A_{g} \oplus A_{u} \oplus 2B_{g} \oplus B_{u} \oplus E_{g} \oplus 2E_{u}}$ \\ 
		Br2 & 4e & $C_{4}$ & $\uni{A_{g} \oplus A_{u} \oplus E_{g} \oplus E_{u}}$ \\ \hline
		&  &  & $\uni{\Gamma_{Total} = 3A_{g} \oplus 5A_{u} \oplus 3B_{g} \oplus B_{u} \oplus 3E_{g} \oplus 6E_{u}}$ \\
		&  &  & $\uni{\Gamma_{acoustic} = A_{u} \oplus E_{u}}$ \\
		&  &  & $\uni{\Gamma_{IR} = 4A_{u} \oplus 5E_{u}}$ \\
		&  &  & $\uni{\Gamma_{Raman} = 3A_{g} \oplus 3B_{g} \oplus 3E_{g}}$ \\
		&  &  & $\uni{\Gamma_{silent} = B_{u}}$ \\ \hline \hline
	\end{tabular}
	\caption{Teoria de grupo da perovskita dupla $\uni{Cs_2 Ag Bi Br_6}$ na fase tetragonal. $\uni{a = 7,8794\AA}$ e $\uni{c = 11,3236\AA}$ em $30 \uni{K}$.}
	\label{tab_tetragonal}
\end{table}
 
\begin{figure}[h!]
    \centering
    \includegraphics[width=8cm]{../Imagens/PDF_Cs2AgBiBr6_raman_temperature}
    \caption{Espectros Raman em função da temperatura para a perovskita dupla $\uni{Cs_2 Ag Bi Br_6}$. Fase tetragonal $I4/m$, ($C_{4h}^5$, \#87) em preto, espectro na temperatura de transição de $122 \suni{K}$ destacado em vermelho e fase cúbica $Fm \bar{3}m$, ($C_{h}^5$, \#225) em azul.}
    \label{raman_temperature}
\end{figure}

\begin{figure}[h!]
    \centering
	\includegraphics[width=7.4cm]{../Imagens/PDF_Cs2AgBiBr6_peak_center}
	\includegraphics[width=7cm]{../Imagens/peak_fwhm}
    \caption{Posição dos centros dos picos (azul) e largura-a-meia-altura (vermelho) dos modos Raman em função da temperatura. Evidenciando uma forte mudança de comportamento abaixo da temperatura crítica ($T_c =$ 122 K), mas principalmente no modo T$_{2g}$ em torno de 73 $\uni{cm^{-1}}$.}
    \label{peak_center}
\end{figure}

\begin{figure}[ht] \centering
	\includegraphics[width=49mm]{../Imagens/PNG_300K}
	\includegraphics[width=49mm]{../Imagens/PNG_125K}
	\includegraphics[width=49mm]{../Imagens/PNG_120K} \\
	\includegraphics[width=49mm]{../Imagens/PNG_113K}
	\includegraphics[width=49mm]{../Imagens/PNG_110K}
	\includegraphics[width=49mm]{../Imagens/PNG_103K}
	\caption{Termomicroscopia do $\uni{Cs_2 Ag Bi Br_6}$. Mudança de birefringência é observada em temperaturas abaixo de 125 K.}
	\label{termomicroscopia}
\end{figure}

\begin{table}[h!] \centering
	\begin{tabular}{lcl} \hline
		\multicolumn{3}{l}{$\boldsymbol{\uni{CaCl_2}}$ \quad $T_c =$ 490 K} \\ \hline
		\multicolumn{3}{l}{Redução da temperatura} \\ \hline
		Tetragonal & $\rightarrow$ & Ortorrômbica \\ 
		Paraelastica & $\rightarrow$ & Ferroelastica \\ \hline
		\multicolumn{3}{l}{Mudança da simetria do fônon} \\ \hline
		B$_{1g}$ & $\rightarrow$ & A$_{g}$ \\ \hline
	\end{tabular}

	\begin{tabular}{lcl} \hline
		\multicolumn{3}{l}{$\boldsymbol{\uni{BiVO_4}}$ \quad $T_c =$ 528 K} \\ \hline 
		\multicolumn{3}{l}{Redução da temperatura} \\ \hline
		Tetragonal & $\rightarrow$ & Monoclínica \\ 
		Paraelastica & $\rightarrow$ & Ferroelastica \\ \hline
		\multicolumn{3}{l}{Mudança da simetria do fônon} \\ \hline
		B$_{g}$ & $\rightarrow$ & A$_{g}$ \\ \hline
	\end{tabular}

	\begin{tabular}{lcl} \hline
		\multicolumn{3}{l}{$\boldsymbol{\uni{LaP_5O_{14}}}$ \quad $T_c =$ 397 K} \\ \hline 
		\multicolumn{3}{l}{Redução da temperatura} \\ \hline
		Ortorrômbica & $\rightarrow$ & Monoclínica \\ 
		Paraelastica & $\rightarrow$ & Ferroelastica \\ \hline
		\multicolumn{3}{l}{Mudança da simetria do fônon} \\ \hline
		B$_{2g}$ & $\rightarrow$ & A$_{g}$ \\ \hline
	\end{tabular}

	\begin{tabular}{lcl} \hline
		\multicolumn{3}{l}{$\boldsymbol{\uni{GeO_2}}$ \quad $P_c =$ 27 GPa} \\ \hline
		\multicolumn{3}{l}{Aumento da pressão} \\ \hline
		Tetragonal & $\rightarrow$ & Ortorrômbica \\ 
		Paraelastica & $\rightarrow$ & Ferroelastica \\ \hline
		\multicolumn{3}{l}{Mudança da simetria do fônon} \\ \hline
		B$_{2g}$ & $\rightarrow$ & A$_{g}$ \\ \hline
	\end{tabular}
	\caption{Transição de fase ferroelástica. Comparativos das características observadas nas amostras de $\uni{CaCl_2}$, $\uni{BiVO_4}$, $\uni{LaP_5O_{14}}$ e $\uni{GeO_2}$ das referências \cite{unruh1992, pinczuk1977, errandonea1981, haines1998}.}
	\label{t_ferro}
\end{table}

\begin{equation}
	S(\omega) = \frac{\atp{\omega_L - \omega_0}^4 \atc{\displaystyle \frac{1}{\displaystyle \exp{\atp{\frac{h \omega c}{k T}}} - 1} + 1} \; I_0 \; \Gamma \; \omega}{\atp{\omega^2 - \omega_0^2} + \Gamma^2 \; \omega^2}
\end{equation}

\begin{equation}
	\omega^2 = \omega_0 + \alpha \atp{T - T_c}
\end{equation}

\begin{equation}
	\omega_s^2 = \omega_p + \alpha_p \atp{T - T_p} \qquad T \geq T_c
\end{equation}
\begin{equation}
	\omega_s^2 = \omega_f + \alpha_f \atp{T - T_f} \qquad T \leq T_c
\end{equation}

\begin{equation}
	F = F_0 + \atp{\frac{a}{2}} x_1^2 - d \; x_1 \; Q_1 + \atc{\frac{\alpha_p\atp{T - T_p}}{2}} Q_1^2
\end{equation}

\begin{equation}
	T_c = T_p + \frac{d^2}{\alpha_p \; a} %\quad\rightarrow\quad \mbox{ferroelastic transition}
	\label{condicao}
\end{equation}

\begin{figure}[h!]
    \centering
    \includegraphics[width=8cm]{../Imagens/PNG_Cs2AgBiBr6_w_w2}
    \caption{Quadrado do modo soft em proporcionalidade com a temperatura}
    \label{raman_w_w2}
\end{figure}

\begin{table}[h!] \centering
	\begin{tabular}{lcl} \hline
		\multicolumn{3}{l}{$\boldsymbol{\uni{Cs_2 Ag Bi Br_6}}$ \quad $T_c =$ 122 K} \\ \hline
		\multicolumn{3}{l}{Redução da temperatura} \\ \hline
		Cubica & $\rightarrow$ & Tetragonal \\ 
		Paraelástica & $\rightarrow$ & ? \\ \hline
		\multicolumn{3}{l}{Mudança da simetria do fônon} \\ \hline
		T$_{2g}$ & $\rightarrow$ & B$_{g} \oplus E_{g}$ \\ \hline
	\end{tabular}
	\caption{Mudança de simetria do fônom anômalo da perovskita $\uni{Cs_2 Ag Bi Br_6}$}
	\label{comp_w_w2}
\end{table}

\begin{equation}
	\Delta_{S} =
	\left(
	\begin{array}{ccc}
		a & 0 & 0 \\
		0 & b & 0 \\
		0 & 0 & b
	\end{array}
	\right)
\label{formaS}
\end{equation}

\begin{equation}
	\Delta_{S'} =
	\left(
	\begin{array}{ccc}
		b & 0 & 0 \\
		0 & a & 0 \\
		0 & 0 & b
	\end{array}
	\right)
\label{formaSl}
\end{equation}

\begin{equation}
	\Delta_{S''} =
	\left(
	\begin{array}{ccc}
		a & 0 & 0 \\
		0 & b & 0 \\
		0 & 0 & b
	\end{array}
	\right)
\label{formaSll}
\end{equation}
		% \chapter{Resultados}

    \section{Aenean Placerat}

    \section{Vivamus Quis Tortor}



		% \chapter{Conclusões}

	\section{Phasellus eu tellus}

		\lipsum[4]

	\section{Quisque ullamcorper}

		\lipsum[2]

	\section{Quisque egestas}

		\lipsum[1]
	\appendix
		% Carraga os apêndices
		\chapter{Apêndice Um}

	\section{Vestibulum}

		\lipsum[3]

		\chapter{Apêndice Dois}

	\section{Phasellus}

		\lipsum[2]
		\chapter{Apêndice Três}

	\section{Pellentesque}

		\lipsum[1]
	\backmatter
		\bibliographystyle{unsrt} % Formatação da bibliografia (ver opções) -> unsrt, plain
        \bibliography{bibliografia}
		\printindex
\end{document}