\chapter{Experimental}

\section{Equações}

$y(x) = ax^2 + bx + c$

$$
	x = \frac{-b \pm \sqrt{b^2 -4ac}}{2a}
$$

\begin{equation}
	x = \frac{-b \pm \sqrt{b^2 -4ac}}{2a}
\end{equation}

% \begin{table}[ht] \centering
% 	\begin{tabular}{cccl} \hline \hline
% 	    \multicolumn{4}{l}{Cúbica - $Fm\bar{3}m$ - $O^5_h$ - $\#225$}
%  	    \\ \hline
% 		Íon & Sítio & Simetria & Contribuição \\ \hline
% 		Cs & 8c & $T_d$ & $\uni{T_{1u} \oplus T_{2g}}$\\
% 		Ag & 4a & $O_h$ & $\uni{T_{1u}}$ \\
% 		Bi & 4b & $O_h$ & $\uni{T_{1u}}$ \\
% 		Br & 24e & $C_{4v}$ & $\uni{A_{1g} \oplus E_{g} \oplus T_{1g} \oplus T_{2g} \oplus 2T_{1u} \oplus T_{2u}}$ \\ \hline
% 		&  &  & $
% 		\uni{\Gamma_{total} = A_{1g} \oplus E_{g} \oplus T_{1g} \oplus 2T_{2g} \oplus 5T_{1u} \oplus T_{2u}} $ \\
% 		&  &  & $\uni{\Gamma_{acoustic} =\uni{T_{1u}}}$ \\
% 		&  &  & $\uni{\Gamma_{IR} = 4\uni{T_{1u}}}$ \\
%  		&  &  & $\uni{\Gamma_{Raman} = A_{1g} \oplus E_{g} \oplus 2T_{2g}}$ \\
%  		&  &  & $\uni{\Gamma_{silent} = T_{1g} \oplus T_{2u}}$ \\ 
%  		\hline \hline
% 	\end{tabular}
% 	\caption{Teoria de grupo da peroviskita dupla $\uni{Cs_2 Ag Bi Br_6}$ na sua fase cúbica. $\uni{a = 11,264\AA}$ em $300 \uni{K}$.}
% 	\label{tab_cubic}
% \end{table}

% $$
% 	\begin{array}{l}
% 		\mbox{Transformação dos} \\
% 		\mbox{modos Raman}
% 	\end{array}
% 	\left\{
% 		\begin{array}{lll}
% 			\uni{A_{1g}} & \rightarrow & \uni{A_{g}} \\
% 			\uni{E_{g}} & \rightarrow & \uni{A_{g}} \oplus \uni{B_{g}} \\
% 			\uni{2T_{2g}} & \rightarrow & 2\atp{\uni{B_{g}} \oplus \uni{E_{g}}} \\
% 			\uni{T_{1g}} & \rightarrow & \uni{A_{g}} \oplus \uni{E_{g}} \\ \hline 
% 			& & \uni{3A_{g} \oplus 3B_{g} \oplus 3E_{g}}
% 		\end{array}
% 	\right.
% 	\label{raman_modes_transform}
% $$

% \begin{equation}
% 	\uni{\Gamma_{Raman}^{30 K} = 3A_{g} \oplus 3B_{g} \oplus 3E_{g}}
% 	\label{gamma_raman_30}
% \end{equation}

% $$
% 	\begin{array}{l}
% 		\mbox{Transformação dos} \\
% 		\mbox{modos infravermelhos}
% 	\end{array}
% 	\left\{
% 		\begin{array}{lll}
% 			\uni{4T_{1u}} & \rightarrow & 4\atp{\uni{A_{u}} \oplus \uni{E_{u}}} \\
% 			\uni{T_{2u}} & \rightarrow & \uni{B_{u}} \oplus \uni{E_{u}} \\ \hline 
% 			& & \uni{4A_{u} \oplus 5E_{u} \oplus \underbrace{\uni{B_{u}}}_{silent}}
% 		\end{array}
% 	\right.
% 	\label{IR_modes_transform}
% $$

% \begin{equation}
% 	\uni{\Gamma_{IR}^{30 K} = 4A_{u} \oplus 5E_{u}}
% 	\label{gamma_ir_30}
% \end{equation}


% \begin{table}[hb] \centering
% 	\begin{tabular}{cccl} \hline \hline
% 	    \multicolumn{4}{l}{Tetragonal - $I4/m$ - $C^5_{4h}$ - $\#87$}
% 	    \\ \hline
% 		Íon & Sítio & Simetria & Contribuição \\ \hline
% 		Cs & 4d & $S_{4}$ & $\uni{A_{u} \oplus B_{g} \oplus E_{g} \oplus E_{u}}$ \\
% 		Ag & 2a & $C_{4h}$ & $\uni{A_{u} \oplus E_{u}}$ \\
% 		Bi & 2b & $C_{4h}$ & $\uni{A_{u} \oplus E_{u}}$ \\
% 		Br1 & 8h & $C_{s}$ & $\uni{2A_{g} \oplus A_{u} \oplus 2B_{g} \oplus B_{u} \oplus E_{g} \oplus 2E_{u}}$ \\ 
% 		Br2 & 4e & $C_{4}$ & $\uni{A_{g} \oplus A_{u} \oplus E_{g} \oplus E_{u}}$ \\ \hline
% 		&  &  & $\uni{\Gamma_{Total} = 3A_{g} \oplus 5A_{u} \oplus 3B_{g} \oplus B_{u} \oplus 3E_{g} \oplus 6E_{u}}$ \\
% 		&  &  & $\uni{\Gamma_{acoustic} = A_{u} \oplus E_{u}}$ \\
% 		&  &  & $\uni{\Gamma_{IR} = 4A_{u} \oplus 5E_{u}}$ \\
% 		&  &  & $\uni{\Gamma_{Raman} = 3A_{g} \oplus 3B_{g} \oplus 3E_{g}}$ \\
% 		&  &  & $\uni{\Gamma_{silent} = B_{u}}$ \\ \hline \hline
% 	\end{tabular}
% 	\caption{Teoria de grupo da perovskita dupla $\uni{Cs_2 Ag Bi Br_6}$ na fase tetragonal. $\uni{a = 7,8794\AA}$ e $\uni{c = 11,3236\AA}$ em $30 \uni{K}$.}
% 	\label{tab_tetragonal}
% \end{table}
 
% \begin{figure}[h!]
%     \centering
%     \includegraphics[width=8cm]{../Imagens/PDF_Cs2AgBiBr6_raman_temperature}
%     \caption{Espectros Raman em função da temperatura para a perovskita dupla $\uni{Cs_2 Ag Bi Br_6}$. Fase tetragonal $I4/m$, ($C_{4h}^5$, \#87) em preto, espectro na temperatura de transição de $122 \suni{K}$ destacado em vermelho e fase cúbica $Fm \bar{3}m$, ($C_{h}^5$, \#225) em azul.}
%     \label{raman_temperature}
% \end{figure}

% \begin{figure}[h!]
%     \centering
% 	\includegraphics[width=7.4cm]{../Imagens/PDF_Cs2AgBiBr6_peak_center}
% 	\includegraphics[width=7cm]{../Imagens/peak_fwhm}
%     \caption{Posição dos centros dos picos (azul) e largura-a-meia-altura (vermelho) dos modos Raman em função da temperatura. Evidenciando uma forte mudança de comportamento abaixo da temperatura crítica ($T_c =$ 122 K), mas principalmente no modo T$_{2g}$ em torno de 73 $\uni{cm^{-1}}$.}
%     \label{peak_center}
% \end{figure}

% \begin{figure}[ht] \centering
% 	\includegraphics[width=49mm]{../Imagens/PNG_300K}
% 	\includegraphics[width=49mm]{../Imagens/PNG_125K}
% 	\includegraphics[width=49mm]{../Imagens/PNG_120K} \\
% 	\includegraphics[width=49mm]{../Imagens/PNG_113K}
% 	\includegraphics[width=49mm]{../Imagens/PNG_110K}
% 	\includegraphics[width=49mm]{../Imagens/PNG_103K}
% 	\caption{Termomicroscopia do $\uni{Cs_2 Ag Bi Br_6}$. Mudança de birefringência é observada em temperaturas abaixo de 125 K.}
% 	\label{termomicroscopia}
% \end{figure}

% \begin{table}[h!] \centering
% 	\begin{tabular}{lcl} \hline
% 		\multicolumn{3}{l}{$\boldsymbol{\uni{CaCl_2}}$ \quad $T_c =$ 490 K} \\ \hline
% 		\multicolumn{3}{l}{Redução da temperatura} \\ \hline
% 		Tetragonal & $\rightarrow$ & Ortorrômbica \\ 
% 		Paraelastica & $\rightarrow$ & Ferroelastica \\ \hline
% 		\multicolumn{3}{l}{Mudança da simetria do fônon} \\ \hline
% 		B$_{1g}$ & $\rightarrow$ & A$_{g}$ \\ \hline
% 	\end{tabular}

% 	\begin{tabular}{lcl} \hline
% 		\multicolumn{3}{l}{$\boldsymbol{\uni{BiVO_4}}$ \quad $T_c =$ 528 K} \\ \hline 
% 		\multicolumn{3}{l}{Redução da temperatura} \\ \hline
% 		Tetragonal & $\rightarrow$ & Monoclínica \\ 
% 		Paraelastica & $\rightarrow$ & Ferroelastica \\ \hline
% 		\multicolumn{3}{l}{Mudança da simetria do fônon} \\ \hline
% 		B$_{g}$ & $\rightarrow$ & A$_{g}$ \\ \hline
% 	\end{tabular}

% 	\begin{tabular}{lcl} \hline
% 		\multicolumn{3}{l}{$\boldsymbol{\uni{LaP_5O_{14}}}$ \quad $T_c =$ 397 K} \\ \hline 
% 		\multicolumn{3}{l}{Redução da temperatura} \\ \hline
% 		Ortorrômbica & $\rightarrow$ & Monoclínica \\ 
% 		Paraelastica & $\rightarrow$ & Ferroelastica \\ \hline
% 		\multicolumn{3}{l}{Mudança da simetria do fônon} \\ \hline
% 		B$_{2g}$ & $\rightarrow$ & A$_{g}$ \\ \hline
% 	\end{tabular}

% 	\begin{tabular}{lcl} \hline
% 		\multicolumn{3}{l}{$\boldsymbol{\uni{GeO_2}}$ \quad $P_c =$ 27 GPa} \\ \hline
% 		\multicolumn{3}{l}{Aumento da pressão} \\ \hline
% 		Tetragonal & $\rightarrow$ & Ortorrômbica \\ 
% 		Paraelastica & $\rightarrow$ & Ferroelastica \\ \hline
% 		\multicolumn{3}{l}{Mudança da simetria do fônon} \\ \hline
% 		B$_{2g}$ & $\rightarrow$ & A$_{g}$ \\ \hline
% 	\end{tabular}
% 	\caption{Transição de fase ferroelástica. Comparativos das características observadas nas amostras de $\uni{CaCl_2}$, $\uni{BiVO_4}$, $\uni{LaP_5O_{14}}$ e $\uni{GeO_2}$ das referências \cite{unruh1992, pinczuk1977, errandonea1981, haines1998}.}
% 	\label{t_ferro}
% \end{table}

% \begin{equation}
% 	S(\omega) = \frac{\atp{\omega_L - \omega_0}^4 \atc{\displaystyle \frac{1}{\displaystyle \exp{\atp{\frac{h \omega c}{k T}}} - 1} + 1} \; I_0 \; \Gamma \; \omega}{\atp{\omega^2 - \omega_0^2} + \Gamma^2 \; \omega^2}
% \end{equation}

% \begin{equation}
% 	\omega^2 = \omega_0 + \alpha \atp{T - T_c}
% \end{equation}

% \begin{equation}
% 	\omega_s^2 = \omega_p + \alpha_p \atp{T - T_p} \qquad T \geq T_c
% \end{equation}
% \begin{equation}
% 	\omega_s^2 = \omega_f + \alpha_f \atp{T - T_f} \qquad T \leq T_c
% \end{equation}

% \begin{equation}
% 	F = F_0 + \atp{\frac{a}{2}} x_1^2 - d \; x_1 \; Q_1 + \atc{\frac{\alpha_p\atp{T - T_p}}{2}} Q_1^2
% \end{equation}

% \begin{equation}
% 	T_c = T_p + \frac{d^2}{\alpha_p \; a} %\quad\rightarrow\quad \mbox{ferroelastic transition}
% 	\label{condicao}
% \end{equation}

% \begin{figure}[h!]
%     \centering
%     \includegraphics[width=8cm]{../Imagens/PNG_Cs2AgBiBr6_w_w2}
%     \caption{Quadrado do modo soft em proporcionalidade com a temperatura}
%     \label{raman_w_w2}
% \end{figure}

% \begin{table}[h!] \centering
% 	\begin{tabular}{lcl} \hline
% 		\multicolumn{3}{l}{$\boldsymbol{\uni{Cs_2 Ag Bi Br_6}}$ \quad $T_c =$ 122 K} \\ \hline
% 		\multicolumn{3}{l}{Redução da temperatura} \\ \hline
% 		Cubica & $\rightarrow$ & Tetragonal \\ 
% 		Paraelástica & $\rightarrow$ & ? \\ \hline
% 		\multicolumn{3}{l}{Mudança da simetria do fônon} \\ \hline
% 		T$_{2g}$ & $\rightarrow$ & B$_{g} \oplus E_{g}$ \\ \hline
% 	\end{tabular}
% 	\caption{Mudança de simetria do fônom anômalo da perovskita $\uni{Cs_2 Ag Bi Br_6}$}
% 	\label{comp_w_w2}
% \end{table}

% \begin{equation}
% 	\Delta_{S} =
% 	\left(
% 	\begin{array}{ccc}
% 		a & 0 & 0 \\
% 		0 & b & 0 \\
% 		0 & 0 & b
% 	\end{array}
% 	\right)
% \label{formaS}
% \end{equation}

% \begin{equation}
% 	\Delta_{S'} =
% 	\left(
% 	\begin{array}{ccc}
% 		b & 0 & 0 \\
% 		0 & a & 0 \\
% 		0 & 0 & b
% 	\end{array}
% 	\right)
% \label{formaSl}
% \end{equation}

% \begin{equation}
% 	\Delta_{S''} =
% 	\left(
% 	\begin{array}{ccc}
% 		a & 0 & 0 \\
% 		0 & b & 0 \\
% 		0 & 0 & b
% 	\end{array}
% 	\right)
% \label{formaSll}
% \end{equation}